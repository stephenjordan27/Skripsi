%versi 2 (8-10-2016) 
\chapter{Pendahuluan}
\label{chap:intro}
   
\section{Latar Belakang}
\label{sec:label}
Perkembangan penggunaan internet dan teknologi informasi mengakibatkan pertumbuhan data yang sangat besar dan terjadi secara terus-menerus sehingga data sulit untuk dikelola, diproses, maupun dianalisis menggunakan teknologi pengolahan data biasa. Data yang terus bertumbuh menyebabkan basis data konvensional menjadi kurang efektif untuk mengolah data. Teknologi saat ini telah menemukan sebuah cara untuk mengurangi biaya penyimpanan dan komputasi data, sehingga kapasitas data dapat ditingkatkan dan data menjadi lebih mudah diolah.

{\it Big data} adalah data dalam jumlah sangat besar dikumpulkan, disimpan, diolah, dan dianalisis agar menghasilkan informasi yang bermanfaat sebagai dasar pengambilan keputusan atau kebijakan. {\it Data mining} adalah teknik ekstraksi informasi terhadap sekumpulan data dalam jumlah besar. {\it Data mining} efektif menggantikan pemrosesan kueri pada basis data dengan data berukuran besar. Masalah yang umum terjadi adalah data yang tersimpan banyak mengandung data yang bersifat privasi sehingga perlu adanya perlindungan privasi pada data yang akan diolah.

Perlindungan privasi dicapai dengan metode enkripsi dan anonimisasi. Enkripsi adalah metode yang memanfaatkan pola atau kunci tertentu. Anonimisasi adalah metode yang menyamarkan satu atau lebih nilai atribut data. Pada kasus tertentu, keamanan enkripsi dapat ditembus melalui penalaran nilai atribut. Penalaran ini sangat berbahaya karena menghubungkan nilai atribut data secara tidak langsung, dapat mengungkapkan entitas pemilik data. Dengan menerapkan konsep anonimisasi diharapkan nilai keterhubungan antar atribut data dapat diperkecil.

Dengan melakukan anonimisasi pada sebagian nilai atribut, bobot informasi yang diperoleh akan semakin kecil. Permasalahan {\it K-anonymity} adalah pencarian solusi untuk menyeimbangkan nilai informasi yang diperoleh dengan nilai informasi yang disamarkan. Permasalahan {\it K-anonymity} diuji dengan pendekatan generalisasi dan supresi. Hasilnya dinilai kurang efektif karena tingginya jumlah informasi yang hilang. Berdasarkan penelitian, permasalahan {\it K-anonymity} tercapai melalui penerapan {\it K-member clustering}. Penerapan {\it K-member clustering} pada algoritma {\it Greedy K-member clustering} dinilai baik karena dapat meminimalkan jumlah informasi yang hilang.

Spark adalah {\it framework} yang tepat untuk memproses data dengan ukuran yang relatif besar seperti {\it big data}, dengan membagi data tersebut ke sistem terdistribusi. Penggunaan Spark menggeser penggunaan Map Reduce pada Hadoop yang dinilai cukup lambat. Kelebihannya adalah Spark memiliki proses komputasi yang lebih cepat karena sebagian besar pemrosesan Spark berada pada RAM. Selain itu, Spark mampu melakukan pemrosesan {\it data mining} menggunakan {\it library} tambahan Spark MLlib. Kekurangannya adalah Spark masih tetap bergantung pada mekanisme penyimpanan Hadoop, agar hasil pemrosesan data dapat tersimpan di dalam {\it hardisk} komputer.

Pada skripsi ini, akan dibuat sebuah perangkat lunak yang dapat memproses data semi terstruktur menjadi data anonimisasi menggunakan konsep {\it K-anonimity}. Perangkat lunak ini berjalan di atas Spark untuk memudahkan proses anonimisasi pada lingkungan {\it big data}. Algoritma {\it Greedy K-member clustering} dinilai tepat untuk melakukan anonimisasi data karena meminimalkan jumlah informasi yang hilang saat proses {\it data mining} di penelitian sebelumnya. Penelitian ini bertujuan membandingkan hasil {\it data mining} sebelum dan setelah dilakukan anonimisasi.

\section{Rumusan Masalah}
\label{sec:rumusan}
Berdasarkan deskripsi diatas, rumusan masalah pada skripsi ini adalah sebagai berikut:
\begin{enumerate}
\item Bagaimana cara kerja algoritma {\it Greedy K-member clustering} ?
\item Bagaimana implementasi algoritma {\it Greedy K-member clustering} pada Spark?
\item Bagaimana hasil {\it data mining} sebelum dan setelah dilakukan anonimisasi?
\end{enumerate}

\section{Tujuan}
\label{sec:tujuan}
Berdasarkan rumusan masalah di atas, tujuan dari skripsi ini adalah sebagai berikut:
\begin{enumerate}
\item Mempelajari cara kerja algoritma {\it Greedy K-member clustering}.
\item Mengimplementasikan algoritma {\it Greedy K-member clustering } pada Spark.
\item Menganalisis hasil {\it data mining} sebelum dan setelah dilakukan anonimisasi.
\end{enumerate}

\section{Batasan Masalah}
\label{sec:batasan}
Batasan masalah pada pengerjaan skripsi ini adalah sebagai berikut:
\begin{enumerate}
\item Perangkat lunak dapat berjalan diatas Spark.
\item Perangkat lunak dapat menerapkan algoritma {\it Greedy K-member clustering}.
\item Perangkat lunak dapat diimplementasikan menggunakan {\it library} Scala-swing.
\item Perangkat lunak hanya menerima input data semi terstruktur CSV dan XML.
\item Menggunakan teknik {\it data mining} yang tersedia pada {\it library} Spark MLlib.
\item Membandingkan hasil {\it data mining} sebelum dan setelah dilakukan anonimisasi.
\end{enumerate}

\section{Metodologi}
\label{sec:metlit}
Bagian-bagian pengerjaan skripsi ini adalah sebagai berikut:
\begin{enumerate}
\item Mempelajari dasar-dasar privasi data.
\item Mempelajari konsep {\it K-anonimity} pada algoritma {\it Greedy K-member clustering}.
\item Mempelajari teknik-teknik dasar {\it data mining}.
\item Mempelajari konsep Hadoop, Spark, dan Spark MLlib.
\item Mempelajari bahasa pemrograman Scala pada Spark.
\item Melakukan analisis masalah dan mengumpulkan data studi kasus.
\item Mengimplementasikan algoritma {\it Greedy K-member clustering } pada Spark.
\item Mengimplementasikan tampilan perangkat lunak menggunakan {\it library} Scala-swing.
\item Mengimplementasikan teknik {\it data mining} menggunakan {\it library} Spark MLlib.
\item Melakukan pengujian fungsional dan experimental.
\item Melakukan analisis hasil {\it data mining} sebelum dan setelah dilakukan anonimisasi.
\item Menarik kesimpulan berdasarkan hasil eksperimen yang telah dilakukan.
\end{enumerate}

\section{Sistematika Pembahasan}
\label{sec:sispem}
Pengerjaan skripsi ini tersusun atas enam bab sebagai berikut:
\begin{itemize}
\item Bab 1 Pendahuluan\\
Berisi latar belakang, rumusan masalah, tujuan, batasan masalah, metodologi penelitian, dan sistematika pembahasan.
\item Bab 2 Dasar Teori\\
Berisi dasar teori tentang dasar-dasar dari privasi, {\it K-anonimity} berbasis {\it clustering}, dan metode pada teknik {\it data mining}.
\item Bab 3 Analisis Masalah\\
Berisi analisis masalah, studi kasus, diagram alian proses.
\item Bab 4 Perancangan \\
Berisi perancangan antarmuka dan diagram kelas.
\item Bab 5 Implementasi dan Pengujian\\
Berisi implementasi perangkat lunak, pengujian fungsional, pengujian eksperimental, dan melakukan analisis terhadap hasil pengujian.
\item Bab 6 Kesimpulan dan Saran\\
Berisi kesimpulan penelitian dan saran untuk penelitian selanjutnya.
\end{itemize}
