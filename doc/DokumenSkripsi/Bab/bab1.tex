%versi 2 (8-10-2016) 
\chapter{Pendahuluan}
\label{chap:intro}
   
\section{Latar Belakang}
\label{sec:label}
Berkembangnya penggunaan sistem informasi di jaman sekarang mengakibatkan data dihasilkan dalam jumlah yang sangat banyak. Data yang jumlahnya sangat banyak ini dikumpulkan dan disimpan dalam tabel basis data untuk keperluan analisis data di masa yang akan datang. Data yang dikumpulkan secara terus-menerus apabila tidak dilakukan analisis secara berkala, maka suatu saat nanti jika proses analisis data dilakukan, proses analisis tersebut berlangsung sangat lama karena ukuran datanya sudah terlanjur besar. Data yang terus bertumbuh menyebabkan basis data konvensional menjadi kurang efektif untuk mengolah data. Teknologi \textit{big data} digunakan untuk mengurangi biaya penyimpanan dan komputasi data, sehingga kapasitas data dapat ditingkatkan dan data berukuran besar menjadi lebih mudah untuk diolah.

{\it Big data} adalah kumpulan data dalam jumlah data yang sangat besar disimpan, diolah, dan dianalisis untuk menghasilkan informasi yang bermanfaat sebagai dasar pengambilan keputusan atau kebijakan. Karena \textit{big data} memiliki ukuran data yang besar, sehingga untuk melakukan analisis pada \textit{big data}, data yang sudah terkumpul akan dibagi ke beberapa komputer untuk diolah secara paralel. Konsep ini disebut sistem terdistribusi. Sistem terdistribusi adalah solusi dari pengolahan \textit{big data} karena terbukti dapat mengurangi biaya penyimpanan dan komputasi data karena dilakukan secara paralel. Untuk melakukan analisis data, diperlukan teknik khusus untuk mencari tahu pola apa saja yang terbentuk dari sekumpulan data yang telah dikumpulkan. Oleh karena itu, diperlukan teknik \textit{data mining} untuk melakukan analisis data.


{\it Data mining} adalah teknik yang diciptakan untuk melihat pola yang terbentuk dari sekumpulan data yang telah terkumpul dalam jumlah yang besar. {\it Data mining} terbukti efektif untuk menggantikan pemrosesan kueri pada basis data konvesional dalam kasus analisis \textit{big data}, karena basis data konvesional tidak menerapkan konsep sistem terdistribusi sehingga waktu komputasinya sangat lambat. Pemodelan data mining nantinya akan dijalankan pada sistem terdistribusi, sehingga waktu komputasi untuk analisis \textit{big data} dapat diminimalkan. Hasil \textit{data mining} nantinya akan dipakai untuk berbagai macam kebutuhan. Biasanya sebuah perusahaan akan meminta data dari perusahaan lain untuk kebutuhan analisis. Masalah yang umum terjadi adalah data hasil pengolahan \textit{data mining} banyak mengandung data yang bersifat privasi sehingga perlu adanya cara untuk menjamin perlindungan privasi pada data yang akan didistribusikan.

Perlindungan privasi untuk distribusi data dapat dicapai dengan menggunakan metode enkripsi dan anonimisasi pada hasil pengolahan data mining. Enkripsi adalah metode yang memanfaatkan pola atau kunci tertentu untuk melindungi data yang sifatnya sensitif. Anonimisasi adalah metode yang menyamarkan satu atau lebih nilai atribut data agar data seseorang tidak dapat saling dibedakan dengan data lainnya. Salah satu kekurangan dari metode enkripsi dibandingkan metode anonimisasi adalah keamanan enkripsi dapat diretas melalui penalaran hubungan nilai atribut yang unik untuk setiap baris data. Penalaran ini dicapai dengan menggabungkan seluruh nilai atribut yang unik pada masing-masing baris data untuk membentuk sebuah pola kelompok data. dPenalaran ini sangat berbahaya karena menghubungkan nilai atribut data yang secara tidak langsung dapat mengungkapkan entitas pemilik data. Dengan menerapkan konsep  anonimisasi diharapkan nilai keterhubungan antar atribut data diperkecil sehingga privasi dapat terlindungi

Dengan melakukan anonimisasi pada sebagian nilai atribut, bobot informasi yang diperoleh akan semakin kecil. Semakin kecil bobot informasi yang diperoleh maka pola untuk membentuk kelompok entitas data semakin kecil sehingga perlindungan privasi akan semakin aman. Akan tetapi dengan semakin kecil bobot informasi yang diperoleh maka nilai akurasi yang dihasilkan oleh metode anonimisasi akan semakin kecil. Oleh karena itu diperlukan cara untuk menyeimbangkan keamanan dan nilai akurasi informasi. Permasalahan {\it k-anonymity} adalah pencarian solusi untuk menyeimbangkan nilai akurasi informasi yang diperoleh dengan nilai informasi yang dilindungi. Permasalahan {\it k-anonymity} diuji dengan pendekatan generalisasi dan supresi. Hasilnya dinilai kurang efektif karena tingginya jumlah informasi yang hilang. Berdasarkan penelitian, permasalahan {\it k-anonymity} tercapai melalui penerapan {\it k-member clustering}. Penerapan {\it k-member clustering} pada algoritma {\it Greedy k-member clustering} dinilai baik karena dapat meminimalkan jumlah informasi yang hilang. Agar algoritma \textit{Greedy k-member clustering} dapat berjalan dengan waktu pemrosesan yang tidak terlalu lama, maka akan digunakan \textit{framework} Spark untuk mengatasi permasalahan \textit{big data}.

Spark adalah {\it framework} yang tepat untuk melakukan proses anonimisasi data pada lingkungan \textit{big data}, karena pekerjaan pengolahan data yang besar dapat dibagi ke beberapa komputer pada sistem terdistribusi. Penggunaan Spark dipilih karena Hadoop memiliki waktu pemrosesan \textit{big data} yang lebih lama dari Spark dengan melakukan komputasi pada \textit{hardisk}, sedangkan Spark melakukan komputasi pada memori. Selain itu Spark memiliki jenis \textit{library} yang lebih beragam dibandingkan dengan Hadoop. Spark  mampu melakukan pemrosesan teknik {\it data mining} pada lingkungan \textit{big data} menggunakan {\it library} tambahan yaitu Spark MLlib. Spark MLlib menfasilitasi pemodelan \textit{data mining} yaitu klasifikasi dan pengelompokan/\textit{clustering}. Kekurangan dari Spark adalah tidak mempunyai penyimpanan yang tetap, sehingga membutuhkan  mekanisme penyimpanan Hadoop, agar hasil pemrosesan data dapat tersimpan di dalam {\it hardisk} komputer.

Pada skripsi ini, akan dibuat dua jenis perangkat lunak yaitu perangkat lunak untuk anonimisasi data dan perangkat lunak untuk analisis data. Perangkat lunak anonimisasi data menggunakan konsep {\it k-anonimity} dengan implementasi algoritma \textit{Greedy k-member clustering} agar sebuah data tidak dapat dibedakan dengan $k-1$ data lainnya. Perangkat lunak anonimisasi data dibuat dengan bahasa Scala dan berjalan di atas Spark untuk meminimalkan waktu komputasi pada proses anonimisasi di lingkungan {\it big data}. Algoritma {\it Greedy k-member clustering} dinilai tepat untuk melakukan pengelompokan data karena meminimalkan jumlah informasi yang hilang saat proses {\it data mining} yang terbukti pada penelitian sebelumnya. Kedua jenis perangkat lunak ini menerima data input dalam format CSV. Untuk tampilannya, kedua jenis perangkat lunak akan dibuat menggunakan GUI dari \textit{library} Scala-swing. Penelitian ini memiliki tujuan utama yaitu membandingkan nilai akurasi dari hasil {\it data mining} sebelum dan setelah dilakukan anonimisasi.

\section{Rumusan Masalah}
\label{sec:rumusan}
Berdasarkan latar belakang di atas, rumusan masalah pada skripsi ini adalah sebagai berikut:
\begin{enumerate}
\item Bagaimana cara kerja algoritma {\it Greedy k-member clustering} ?
\item Bagaimana implementasi algoritma {\it Greedy k-member clustering} pada Spark?
\item Bagaimana hasil {\it data mining} sebelum dan setelah dilakukan anonimisasi?
\end{enumerate}

\newpage
\section{Tujuan}
\label{sec:tujuan}
Berdasarkan rumusan masalah di atas, tujuan dari skripsi ini adalah sebagai berikut:
\begin{enumerate}
\item Mempelajari cara kerja algoritma {\it Greedy k-member clustering}.
\item Mengimplementasikan algoritma {\it Greedy k-member clustering } pada Spark.
\item Menganalisis hasil {\it data mining} sebelum dan setelah dilakukan anonimisasi.
\end{enumerate}

\section{Batasan Masalah}
\label{sec:batasan}
Batasan masalah pada pengerjaan skripsi ini adalah sebagai berikut:
\begin{enumerate}
\item Perangkat lunak dapat berjalan diatas Spark.
\item Perangkat lunak dapat menerapkan algoritma {\it Greedy k-member clustering}.
\item Perangkat lunak dapat diimplementasikan menggunakan {\it library} Scala-swing.
\item Perangkat lunak hanya menerima input data semi terstruktur CSV dan XML.
\item Menggunakan teknik {\it data mining} yang tersedia pada {\it library} Spark MLlib.
\item Membandingkan hasil {\it data mining} sebelum dan setelah dilakukan anonimisasi.
\end{enumerate}

\section{Metodologi}
\label{sec:metlit}
Bagian-bagian pengerjaan skripsi ini adalah sebagai berikut:
\begin{enumerate}
\item Mempelajari dasar-dasar privasi data.
\item Mempelajari konsep {\it k-anonimity} pada algoritma {\it Greedy k-member clustering}.
\item Mempelajari teknik-teknik dasar {\it data mining}.
\item Mempelajari konsep Hadoop, Spark, dan Spark MLlib.
\item Mempelajari bahasa pemrograman Scala pada Spark.
\item Melakukan analisis masalah dan mengumpulkan data studi kasus.
\item Mengimplementasikan algoritma {\it Greedy k-member clustering } pada Spark.
\item Mengimplementasikan tampilan perangkat lunak menggunakan {\it library} Scala-swing.
\item Mengimplementasikan teknik {\it data mining} menggunakan {\it library} Spark MLlib.
\item Melakukan pengujian fungsional dan experimental.
\item Melakukan analisis hasil {\it data mining} sebelum dan setelah dilakukan anonimisasi.
\item Menarik kesimpulan berdasarkan hasil eksperimen yang telah dilakukan.
\end{enumerate}

\newpage
\section{Sistematika Pembahasan}
\label{sec:sispem}
Pengerjaan skripsi ini tersusun atas enam bab sebagai berikut:
\begin{itemize}
\item Bab 1 Pendahuluan\\
Berisi latar belakang, rumusan masalah, tujuan, batasan masalah, metodologi penelitian, dan sistematika pembahasan.
\item Bab 2 Dasar Teori\\
Berisi dasar teori tentang dasar-dasar dari privasi, {\it k-anonimity} berbasis {\it clustering}, dan metode pada teknik {\it data mining}.
\item Bab 3 Analisis Masalah\\
Berisi analisis masalah, studi kasus, diagram alian proses.
\item Bab 4 Perancangan \\
Berisi perancangan antarmuka dan diagram kelas.
\item Bab 5 Implementasi dan Pengujian\\
Berisi implementasi perangkat lunak, pengujian fungsional, pengujian eksperimental, dan melakukan analisis terhadap hasil pengujian.
\item Bab 6 Kesimpulan dan Saran\\
Berisi kesimpulan penelitian dan saran untuk penelitian selanjutnya.
\end{itemize}
