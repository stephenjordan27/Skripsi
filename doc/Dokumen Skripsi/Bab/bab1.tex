%versi 2 (8-10-2016) 
\chapter{Pendahuluan}
\label{chap:intro}
   
\section{Latar Belakang}
\label{sec:label}


Untuk menghasilkan keputusan bisnis yang berkualitas, perusahaan memerlukan pengetahuan  yang diperoleh melalui proses analisis data yang sangat banyak agar mendapatkan hasil yang valid. Biasanya data dikumpulkan fari berbagai platform seperti e-commerce, sosial media, maupun produk transaksi digital lainnya. Data yang disimpan dapat berupa data terstruktur, semi terstruktur, dan tidak terstruktur. Data testruktur adalah data yang disimpan dalam bentuk tabel, baris dan kolom, contohnya {\it file} excel atau spreadsheet. Data semi-terstuktur adalah data yang disimpan dengan format tertentu, contohnya {\it file} CSV, JSON. Data tidak terstruktur adalah data yang tidak memiliki format penyimpanan nilai, contohnya {\it file} text. Data-data ini nantinya disimpan menggunakan teknologi basis data untuk di analisis lebih lanjut. Data yang dikumpulkan dapat memiliki kapasitas penyimpanan yang besar, sehingga proses analisis data menjadi sangat lambat. Dampak yang ditimbulkan dari pertumbuhan data menyebabkan basis data konvensional menjadi kurang efektif untuk pengolahan data. Oleh karena itu, teknologi \textit{big data} digunakan untuk mengurangi biaya yang dikeluarkan untuk penyimpanan dan komputasi data, sehingga kapasitas data dapat ditambah dan data berukuran besar dapat lebih mudah untuk diolah.

Menurut Gartner, salah satu perusahaan riset teknologi informasi di Amerika Serikat, {\it big data} adalah adalah aset informasi bervolume tinggi, cepat, dan beragam yang menuntut bentuk pemrosesan informasi yang hemat biaya dan inovatif untuk meningkatkan wawasan dan pengambilan keputusan\footnote{https://www.gartner.com/en/information-technology/glossary/big-data}. {\it Big data} disimpan, diolah, dan dilakukan analisis agar menghasilkan informasi yang bermanfaat sebagai dasar pengambilan keputusan atau kebijakan yang lebih tepat berdasarkan data sebenarnya. Karena {\it Big data} memiliki ukuran data yang besar, maka proses analisis \textit{big data} harus dilakukan secara paralel. Caranya adalah dengan membagi data ke beberapa komputer untuk diolah masing-masing komputer tersebut. Konsep ini disebut dengan sistem terdistribusi. Sistem terdistribusi adalah solusi pengolahan \textit{big data} karena terbukti dapat mengurangi biaya penyimpanan dan komputasi data dari pemrosesan data secara paralel. 

Untuk melakukan proses analisis data, diperlukan teknik untuk mencari tahu kesamaan sifat yang dimiliki oleh sekumpulan data. Salah satu teknik yang dapat digunakan adalah {\it data mining}. Menurut Gartner, {\it Data mining} adalah proses menemukan korelasi, pola, dan tren yang bermakna menggunakan teknologi pengenalan pola, teknik statistik dan matematika dengan memilah-milah data dalam jumlah besar\footnote{https://www.gartner.com/en/information-technology/glossary/data-mining}. Teknik {\it data mining} dapat membantu proses analisis data pada lingkungan \textit{big data}. Pemodelan data mining untuk \textit{big data} dijalankan pada sistem terdistribusi, sehingga waktu komputasi dapat diminimalkan. Hasil \textit{data mining} dapat menimbulkan masalah privasi.  Privasi adalah hak seseorang untuk memiliki kendali atas bagaimana informasi pribadi dikumpulkan dan digunakan\footnote{https://iapp.org/about/what-is-privacy/}. Privasi seseorang dapat terlanggar apabila sebuah perusahaan meminta data transaksi dari perusahaan lain dan secara tidak sengaja mengetahui informasi pribadi seseorang setelah dilakukan teknik \textit{data mining} pada sekumpulan data masih banyak mengandung data privat.  PPDM merupakan bagian dari data mining yang bertanggung jawab atas perlindungan privasi dalam proses data mining. Oleh karena itu, Privacy Preserving Data Mining (PPDM) berperan penting untuk memberi perlindungan privasi dalam proses {\it data mining}.

Konsep PPDM dapat dicapai dengan metode enkripsi dan anonimisasi. Menurut Gartner, enkripsi adalah proses pengkodean aliran bit secara sistematis sebelum dilakukan transmisi sehingga pihak yang tidak berwenang tidak dapat mengetahui arti pesan sebenarnya\footnote{https://www.gartner.com/en/information-technology/glossary/encryption}. Anonimisasi adalah metode yang menyamarkan satu atau lebih nilai atribut data agar data seseorang tidak dapat saling dibedakan dengan data lainnya. Kekurangan metode enkripsi adalah waktu komputasi yang lebih lama untuk membuat kunci dan biasanya data hasil enkripsi memiliki ukuran lebih besar dari data asli. Di satu sisi metode anonimisasi lebih unggul karena tidak perlu membuat kunci untuk menjaga privasi data. Metode anonimisasi dapat diterapkan pada kasus analisis data kartu kredit untuk mencari tahu  karakteristik klien yang berpotensi untuk diberi kartu pinjaman kredit. Data kartu kredit dipilih karena masih memiliki kolom-kolom yang mengandung informasi pribadi seperti gaji, jenis pekerjaan, jenis pendidikan, kategori tempat tinggal, dan status keluarga.

Umumnya data yang disamarkan dengan metode anonimisasi menderita kehilangan informasi yang cukup banyak. Istilah ini lebih sering dikenal dengan {\it total information loss}. Hal ini mengakibatkan data yang telah dianonimisasi memiliki hasil prediksi yang lebih rendah ketika menggunakan model data mining klasifikasi/clustering. Salah satu cara untuk mencegah hal tersebut adalah dengan melakukan pengelompokan data terlebih dahulu sebelum dilakukan anonimisasi. Contoh algoritma pengelompokan data yang ingin diuji adalah greedy k-member clustering. Algoritma ini dipilih karena mencari solusi paling optimal dari seluruh kemungkinan pengelompokan yang ada. Metode anonimisasi yang digunakan adalah k-anonymity. Metode ini menjaga sebuah data tidak dapat dibedakan dengan $k-1$ data lainnya. Karena penelitian berkaitan dengan big data dan membutuhkan implementasi algoritma secara iteratif maka diperlukan teknologi untuk menangani komputasi secara paralel untuk kinerja yang lebih efisien. Spark merupakan pilihan yang tepat karena dapat memanfaatkan komputasi memori sehingga memiliki komputasi yang lebih cepat dalam implementasi algoritma iteratif maupun menganalisis big data. 

Framework Spark dan Hadoop sering digunakan untuk melakukan komputasi pada lingkungan big data. Hadoop adalah framework yang memungkinkan pemrosesan terdistribusi dari kumpulan data besar di seluruh cluster komputer. Hadoop memiliki sistem penyimpanan yang tersebar di beberapa komputer dengan nama HDFS. Spark adalah mesin analitik terpadu untuk pemrosesan data skala besar. Spark cocok digunakan jika membutuhkan implementasi pemrograman iteratif. Penggunaan Spark dipilih karena Hadoop memiliki waktu pemrosesan \textit{big data} yang lebih lama dari Spark karena melakukan komputasi pada \textit{hardisk}, sedangkan Spark dapat melakukan komputasi pada memori. Selain itu Spark memiliki jenis \textit{library} yang lebih beragam dibandingkan dengan Hadoop. Spark mampu melakukan pemrosesan teknik {\it data mining} pada lingkungan \textit{big data} menggunakan {\it library} tambahan yaitu Spark MLlib. Spark MLlib menfasilitasi pemodelan \textit{data mining} yaitu klasifikasi dan pengelompokan/\textit{clustering}. Kekurangan dari Spark adalah tidak memiliki penyimpanan yang tetap, sehingga membutuhkan  mekanisme penyimpanan Hadoop, agar hasil pemrosesan data dapat tersimpan dalam {\it hardisk} komputer.

Pada skripsi ini, akan dibuat tiga jenis perangkat lunak yaitu perangkat lunak eksplorasi, perangkat lunak anonimisasi, dan perangkat lunak pengujian. Perangkat lunak eksplorasi bertujuan mencari nilai unik untuk pembuatan pohon binary tree yang digunakan ketika menghitung jarak terdekat antar data kategorikal. Perangkat lunak anonimisasi bertujuan mengimplementasikan proses pengelompokan data dengan algoritma \textit{Greedy k-member clustering} dan proses anonimisasi data dengan metode {\it k-anonimity}. Perangkat lunak pengujian bertujuan membandingkan kualitas hasil data mining sebelum dan setelah data  dianonimisasi. Ketiga perangkat lunak ini dibuat dengan bahasa Scala, berjalan di atas Spark, dan menggunakan penyimpanan HDFS untuk menyimpan hasil komputasi sementara dari algoritma greedy k-member clustering. Algoritma {\it Greedy k-member clustering} dinilai tepat melakukan pengelompokan data karena terbukti pada penelitian sebelumnya dapat meminimalkan {\it total information loss}. Kedua jenis perangkat lunak ini menerima data input dalam format CSV. Penelitian ini memiliki tujuan utama yaitu membandingkan silhouette score untuk model clustering, tingkat akurasi untuk model klasifkasi, mencari parameter terbaik pada model clustering dan klasifikasi, dan mencari perbedaan hasil clustering dan klasifikasi sebelum dan setelah dilakukan proses anonimisasi data.

\section{Rumusan Masalah}
\label{sec:rumusan}
Berdasarkan latar belakang di atas, rumusan masalah pada skripsi ini adalah sebagai berikut:
\begin{enumerate}
\item Bagaimana cara kerja algoritma {\it Greedy k-member clustering} untuk pengelompokan data?
\item Bagaimana cara kerja algoritma {\it k-anonymity} untuk anonimisasi data?
\item Bagaimana implementasi algoritma {\it Greedy k-member clustering} pada Spark?
\item Bagaimana performa algoritma {\it Greedy k-member clustering} dan {\it k-anonymity} untuk lingkungan big data pada komputer cluster?
\item Bagaimana performa pemodelan data mining clustering (k-means) dan klasifikasi (naive bayes) untuk lingkungan big data pada komputer lokal?
\item Bagaimana kualitas informasi pada data yang telah dikelompokan dengan algoritma {\it Greedy k-member clustering} berdasarkan total information loss?
\item Bagaimana analisis kualitas hasil data mining terhadap pemodelan clustering sebelum dan setelah dilakukan anonimisasi?
\item Bagaimana analisis kualitas hasil data mining terhadap pemodelan klasifikasi sebelum dan setelah dilakukan anonimisasi?
\end{enumerate}

\section{Tujuan}
\label{sec:tujuan}
Berdasarkan rumusan masalah di atas, tujuan dari skripsi ini adalah sebagai berikut:
\begin{enumerate}
\item Mempelajari cara kerja algoritma {\it Greedy k-member clustering} untuk pengelompokan data.
\item Mempelajari cara kerja algoritma {\it k-anonymity} untuk anonimisasi data.
\item Mengimplementasi algoritma {\it Greedy k-member clustering} dan {\it k-anonymity} pada Spark.
\item Menganalisis performa algoritma {\it Greedy k-member clustering} dan {\it k-anonymity} untuk lingkungan big data pada komputer cluster.
\item Menganalisis performa pemodelan data mining clustering (k-means) dan klasifikasi (naive bayes) untuk lingkungan big data pada komputer lokal.
\item Menganalisis kualitas informasi pada data yang telah dikelompokan dengan algoritma {\it Greedy k-member clustering} berdasarkan total information loss.
\item Menganalisis kualitas hasil metode data mining clustering berdasarkan silhouette score, mencari parameter clustering terbaik, dan mencari persentase perbedaan hasil clustering terbaik sebelum dan setelah dilakukan anonimisasi.
\item Menganalisis kualitas hasil metode data mining klasifikasi berdasarkan tingkat akurasi, mencari parameter klasifikasi terbaik, dan mencari persentase perbedaan hasil klasifikasi terbaik sebelum dan setelah dilakukan anonimisasi.
\end{enumerate}

\section{Batasan Masalah}
\label{sec:batasan}
Batasan masalah pada pengerjaan skripsi ini adalah sebagai berikut:
\begin{enumerate}

\item Perangkat lunak hanya menerima masukan dalam format JSON. Format masukan JSON untuk masing-masing perangkat lunak berbeda dan dapat diisi sesuai kebutuhan. Contoh format JSON untuk masing-masing perangkat lunak telah dicatumkan pada Subbab 4.3

\item Perangkat lunak hanya dapat mengeluarkan output dalam forma CSV. Folder penyimpanan output perangkat lunak dapat diubah pada file input JSON.

\item Perangkat lunak anonimisasi dapat melakukan pengelompokan dan anonimisasi data hingga 100.000 data dengan waktu komputasi selama 2 hari. Disarankan untuk menggunakan  jumlah data dibawah 10.000 data untuk mendapat estimasi waktu komputasi kurang dari 3 jam.


\end{enumerate}

\section{Metodologi}
\label{sec:metlit}
Bagian-bagian pengerjaan skripsi ini adalah sebagai berikut:
\begin{enumerate}
\item Mempelajari dasar-dasar privasi data.
\item Mempelajari konsep {\it k-anonimity} pada algoritma {\it Greedy k-member clustering}.
\item Mempelajari teknik-teknik dasar {\it data mining}.
\item Mempelajari konsep Hadoop, Spark, dan Spark MLlib.
\item Mempelajari bahasa pemrograman Scala pada Spark.
\item Melakukan analisis masalah dan mengumpulkan data studi kasus.
\item Mengimplementasikan algoritma {\it Greedy k-member clustering } pada Spark.
\item Mengimplementasikan tampilan perangkat lunak menggunakan {\it library} Scala-swing.
\item Mengimplementasikan teknik {\it data mining} menggunakan {\it library} Spark MLlib.
\item Melakukan pengujian fungsional dan experimental.
\item Melakukan analisis hasil {\it data mining} sebelum dan setelah dilakukan anonimisasi.
\item Menarik kesimpulan berdasarkan hasil eksperimen yang telah dilakukan.
\end{enumerate}

\section{Sistematika Pembahasan}
\label{sec:sispem}
Pengerjaan skripsi ini tersusun atas enam bab sebagai berikut:

\begin{itemize}

\item Bab 1 Pendahuluan\\
Berisi latar belakang, rumusan masalah, tujuan, batasan masalah, metodologi penelitian, dan sistematika pembahasan.

\item Bab 2 Landasan Teori\\
Berisi landasan teori mengenai konsep privasi, teknik \textit{data mining}, \textit{privacy-preserving data mining}, \textit{k-anonymity}, algoritma \textit{greedy k-member clustering}, metrik \textit{distance} dan \textit{information loss}, teknologi \textit{big data}, pemrograman scala, dan format penyimpanan data.

\item Bab 3 Analisis\\
Berisi analisis penelitian mengenai analisis masalah (dataset eksperimen, \textit{personally identifiable information}, perhitungan \textit{distance} dan \textit{information loss}, algoritma \textit{greedy k-member clustering}, \textit{k-anonymity}, \textit{domain generalization hierarchy}), eksplorasi spark (instalasi spark, pembuatan \textit{project} spark, menjalankan program spark), studi kasus (eksperimen  scala, eksperimen spark), dan gambaran umum perangkat lunak (diagram kelas dan diagram aktivitas).

\item Bab 4 Perancangan \\
Berisi perancangan antarmuka perangkat lunak anonimisasi data dan analisis data, diagram kelas lengkap, masukan perangkat lunak anonimisasi data dan analisis data.

\item Bab 5 Implementasi dan Pengujian\\
Berisi implementasi perangkat lunak anonimisasi data dan analisis data, pengujian fungsional, pengujian eksperimental, dan melakukan analisis terhadap hasil pengujian.

\item Bab 6 Kesimpulan dan Saran\\
Berisi kesimpulan penelitian dan saran untuk penelitian selanjutnya.

\end{itemize}
